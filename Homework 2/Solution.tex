\documentclass[11pt]{article}
\usepackage{geometry}
\geometry{a4paper,top=2cm,bottom=2cm,left=2cm,right=2cm}
\usepackage[english]{babel}
\usepackage{graphicx}
\usepackage{fancyhdr}
\usepackage{lineno}
\usepackage{amsmath}
\usepackage{amssymb}
\usepackage{algorithm}
\usepackage[noend]{algpseudocode}
\usepackage {tikz}
\usepackage{tikz-cd}
\usetikzlibrary {positioning}
\usepackage {xcolor}
\usepackage{titlesec}
\usepackage{varwidth}

\makeatletter
\def\BState{\State\hskip-\ALG@thistlm}
\makeatother

\title{\textbf{Algorithm Design - Homework 2} \\ \bigskip \large \textbf{Sapienza University of Rome}}
\date{\textbf{\today}}
\author{\textbf{Marco Costa, 1691388}}
\pagestyle{fancy}
\fancyhead[L]{Marco Costa, 1691388}
\fancyhead[R]{Algorithm Design - Homework 2}

%section, subsection: space - left, before, after
\titlespacing{\section}{0em}{0em}{0em}
\titlespacing{\subsection}{0em}{0.5em}{0em}
\titlespacing{\subsubsection}{0em}{0.5em}{0em}
\setlength\parindent{0pt}

\begin{document}
\maketitle
\newpage

\section*{Exercise 1}
\subsection*{First Problem}
We need to model Michele's problem as a integer linear problem, and relax this to a corresponding LP.
\subsubsection*{Solution}
The problem can be modeled as an \textit{ILP} as follow: \\
\(\left \{ \begin{array}{l}
minimize \quad \sum\limits_{i \in F} \sum\limits_{j \in O}x_{ij}w(i,j)\\
s.t. \qquad \qquad \sum\limits_{j \in O}x_{ij} = 1 , \forall{i \in F}\\
\qquad \qquad \qquad \sum\limits_{i \in F}x_{ij} \leq 1 , \forall{j \in O}\\
\qquad \qquad \qquad x_{ij} \in{\{0,1\}}
\end{array}
\right. \)\\
For the relaxation of this problem to a LP problem, we need simply to change the last condition in $x_{ij} \in [0, 1]$ ($0 \leq x_{ij} \leq 1$).
\subsection*{Second Problem}
Give a polynomial-time algorithm that, from any
given optimal LP-solution, computes an optimal integer assignment, knowing that there is no \textit{integrality gap} and for every fractional LP-solution, there exists an integral
feasible solution with the same cost.
\subsubsection*{Solution}

\newpage

\section*{Exercise 2}
\subsection*{Problem}

\subsubsection*{Solution}

\newpage

\section*{Exercise 3}
\subsection*{Problem}
Given $\alpha \ge 0$ calculate Philip’s payoff at some mixed Nash equilibrium, for $\alpha < 1$ and $\alpha \ge 1$.
\subsubsection*{Solution for $\mathbf{\alpha < 1}$}
Let's define as $q_r$ the probability that the opponent plays R, $q_p$ the probability that the opponent plays P and $q_s = 1 - q_r - q_p$ the probability that the opponent plays S. Now we compute Philip's expected payoffs:
\begin{itemize}
	\item Philip's expected payoff playing R: $\alpha \cdot q_r + (\alpha - 1) \cdot q_p + (\alpha + 1) \cdot (1 - q_r - q_p)$
	\item Philip's expected payoff playing P: $1 \cdot q_r + 0 \cdot q_p -1 \cdot (1 - q_r - q_p) = 2q_r + q_p - 1$
	\item Philip's expected payoff playing S: $-1 \cdot q_r + 1 \cdot q_p + 0 \cdot (1 - q_r - q_p) = -q_r + q_p$
\end{itemize}
Since the strategy must be best responses, the payoffs must be equal, and we obtain: \\
\begin{minipage}[t]{0.63\textwidth}
\(\left \{ \begin{array}{l}
\alpha \cdot q_r + (\alpha - 1) \cdot q_p + (\alpha + 1) \cdot (1 - q_r - q_p) = 2q_r + q_p - 1\\
\alpha \cdot q_r + (\alpha - 1) \cdot q_p + (\alpha + 1) \cdot (1 - q_r - q_p) = -q_r + q_p\\
2q_r + q_p - 1 = -q_r + q_p
\end{array}
\right. \)
\end{minipage}
\begin{minipage}[t]{0.07\textwidth}
	\raisebox{-.5\height}{\scalebox{2}{$\Rightarrow$}}
\end{minipage}
\begin{minipage}[t]{0.3\textwidth}
	\(\left \{ \begin{array}{l}
		q_r = \frac{1}{3} \\
		q_p = \frac{\alpha + 1}{3} \\
		q_s = 1 - q_r - q_p = \frac{1 - \alpha}{3}
	\end{array}
	\right. \)
\end{minipage} \\
Since Philip is mixing in a Nash Equilibrium, the opponent must mix with weights \(\left(\frac{1}{3},\frac{\alpha + 1}{3},\frac{1 - \alpha}{3}\right) \).\\
Now let's do the same for opponent's expected payoffs, defining as $q'_r$, $q'_p$ and $q'_s = 1 - q'_r - q'_p$ the probability that Philip plays R, P or S respectively:
\begin{itemize}
	\item Opponent expected payoff playing R: $0 \cdot q'_r - 1 \cdot q'_p + 1 \cdot (1 - q'_r - q'_p) = -2q'_p - q'_r + 1$
	\item Opponent expected payoff playing S: $1 \cdot q'_r + 0 \cdot q'_p - 1 \cdot (1 - q'_r - q'_p) = 2q'_r + q'_p - 1$
	\item Opponent expected payoff playing P: $-1 \cdot q'_r + 1 \cdot q'_p + 0 \cdot (1 - q'_r - q'_p) = -q'_r + q'_p$
\end{itemize}
Since the strategy must be best responses, the payoffs must be equal, and we obtain: \\
\begin{minipage}[t]{0.4\textwidth}
	\(\left \{ \begin{array}{l}
	-2q'_p - q'_r + 1 = 2q'_r + q'_p - 1 \\
	-2q'_p - q'_r + 1 = -q'_r + q'_p \\
	-2q'_r + q'_p - 1 = -q'_r + q'_p
	\end{array}
	\right. \)
\end{minipage}
\begin{minipage}[t]{0.07\textwidth}
\raisebox{-.5\height}{\scalebox{2}{$\Rightarrow$}}
\end{minipage}
\begin{minipage}[t]{0.3\textwidth}
\(\left \{ \begin{array}{l}
q'_r = \frac{1}{3} \\
q'_p = \frac{1}{3} \\
q'_s = 1 - q_r - q_p = \frac{1}{3}
\end{array}
\right. \)
\end{minipage} \\
Since the opponent is mixing in a Nash Equilibrium, Philip must mix with weights \(\left(\frac{1}{3},\frac{1}{3},\frac{1}{3}\right) \).\\
So the MNE is:
\(\left[\left(\frac{1}{3},\frac{1}{3},\frac{1}{3}\right),\left(\frac{1}{3},\frac{\alpha + 1}{3},\frac{1 - \alpha}{3}\right) \right] \).\\
Finally we can calculate Philip's payoff as follow: \\
$\alpha \cdot q_r \cdot q'_r + (\alpha - 1) \cdot q_p \cdot q'_r + (\alpha + 1) \cdot q_s \cdot q'_r + 1 \cdot q_r \cdot q'_p + 0 \cdot q_p \cdot q'_p + (-1) \cdot q_s \cdot q'_p + (-1) \cdot q_r \cdot q'_s + 1 \cdot q_p \cdot q'_s + 0 \cdot q_s \cdot q'_s = \frac{\alpha}{3}$

\subsubsection*{Solution for $\mathbf{\alpha \ge 1}$}
\begin{minipage}[c]{10cm}
	In this case, looking at the table it's possible to see that simplifications can be made: for Philip, P is dominated by R because of $\alpha \ge 1$, so we can delete the relative row. After that, for the opponent S is dominated by R, so we can delete the relative column. \\
\end{minipage}
\begin{tabular}{ |c|c|c|c| }
\hline
Ph/Op & R & P & S \\
\hline
R &  $\alpha$, 0  &  $(\alpha-1)$, 1 &  $(\alpha+1)$, -1 \\
\hline
P &  1, -1  &  0 , 0 &  -1 , 1 \\
\hline
S &  -1, 1  &  1 , -1 &  0 , 0 \\
\hline
\end{tabular}
Let's define as $q_r$ and $q_p = 1 - q_r$ the probability that the opponent plays R or P respectively, while as $q'_r$ and $q'_s = 1 - q'_r$ the probability that Philip plays R or S respectively. \\
We have the following payoffs:
\begin{tabular}{ |c|c|c| }
	\hline
	 & Philip & Opponent \\
	\hline
	R &  $\alpha \cdot q_r + (\alpha - 1) \cdot (1 - q_r) = q_r + \alpha - 1$  &  $0 \cdot q'_r + 1 \cdot (1 - q'_r) = 1 - q'_r$  \\
	\hline
	P &  none  & $1 \cdot q'_r -1 \cdot (1 - q'_r) = 2q'_r - 1$ \\
	\hline
	S &  $-1 \cdot q_r + 1 \cdot (1 - q_r) = -2q_r + 1$  &  none  \\
	\hline
\end{tabular}
Since the strategy must be best responses, the payoffs of each player must be equal, and we obtain:  $q_r = \frac{2-\alpha}{3}$, $q_p = \frac{\alpha + 1}{3}$, $q'_r = \frac{2}{3}$ and $q'_s = \frac{1}{3}$. \\
So the MNE is: 
\(\left[\left(\frac{2}{3},\frac{1}{3}\right),\left(\frac{2-\alpha}{3},\frac{\alpha + 1}{3}\right) \right] \).\\
Finally we can calculate Philip's payoff as follow: \\ 
$\alpha \cdot q_r \cdot q'_r + (\alpha - 1) \cdot q_p \cdot q'_r + (-1) \cdot q_r \cdot q'_s + 1 \cdot q_p q'_s = \frac{2\alpha-1}{3}$.

\newpage

\section*{Exercise 4}
\subsection*{Problem}

\subsubsection*{Solution}

\newpage

\section*{Exercise 5}
\subsection*{Problem}

\subsubsection*{Solution}


\end{document}
