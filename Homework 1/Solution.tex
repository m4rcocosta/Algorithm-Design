\documentclass[11pt]{article}
\usepackage{geometry}
\geometry{a4paper,top=2cm,bottom=2cm,left=2cm,right=2cm}
\usepackage[english]{babel}
\usepackage{graphicx}
\usepackage{fancyhdr}
\usepackage{lineno}
\usepackage{amsmath}
\usepackage{algorithm}
\usepackage[noend]{algpseudocode}
\usepackage {tikz}
\usetikzlibrary {positioning}
\usepackage {xcolor}

\makeatletter
\def\BState{\State\hskip-\ALG@thistlm}
\makeatother

\title{\textbf{Homework 1 - Algorithm Design} \\ \bigskip \large \textbf{Sapienza University of Rome}}
\date{\textbf{\today}}
\author{\textbf{Marco Costa, 1691388}}
\pagestyle{fancy}
\fancyhead[L]{Marco Costa, 1691388}
\fancyhead[R]{Algorithm Design - Homework 1}

\begin{document}
\maketitle
\newpage

\section*{Exercise 1}
\subsection*{Problem}
\subsubsection*{Solution}
Let's define a matrix M ($k \times n$), in which we have the expected value opening the $i^{th}$ box having $j$ as current best reward ($i \in [0, k-1]$, $j \in [0, n]$)
\begin{algorithm}
	\caption{Populate the matrix}\label{euclid}
	\begin{algorithmic}[1]
		\For{$i$ in $(0, ..., k-1)$}
		\For{$j$ in $(0, ..., n)$}
		\State $M[i,j] \gets -\sum\limits_{h=1}^i c_h + \frac{j}{n+1}(j-1) + \frac{1}{n+1}\sum\limits_{h=j}^n h$
		\EndFor		
		\EndFor
	\end{algorithmic}
\end{algorithm}\\
Now, for each box we can calculate the "\textit{a priori}" expected value as follow: \\ \\
$\mathbf{\bar{v_k} = \sum\limits_{i=0}^n P_{k-1}(i)v_k(i)}$ \\ \\
where $\mathbf{P_{k-1}(i)}$ is the probability of having reward i at least in one of the $\mathit{k-1}$ boxes, while $\mathbf{v_k(i)}$ is the expected value opening the $\mathit{k^{th}}$ box having as current best reward $\mathit{i}$. \\ \\
$\mathbf{P_{k-1}(i) = (\frac{1}{n+1})(\frac{i+1}{n+1})^{k-2}(k-1)}$, \qquad $\mathbf{v_k(i) = M[k,i]}$ \\ \\
So we have: \\ \\
$\mathbf{\bar{v_k} = \sum\limits_{i=0}^n (\frac{1}{n+1})(\frac{i+1}{n+1})^{k-2}(k-1)M[k,i] = \frac{k-1}{n+1} \sum\limits_{i=0}^n (\frac{i+1}{n+1})^{k-2}M[k,i]}$ \\ \\
Now we have to calculate $\mathbf{\bar{v_k}}$ for each box and return the max:
\begin{algorithm}
	\caption{Find expected optimal reward}\label{euclid}
	\begin{algorithmic}[1]
		\State $k_{max} \gets 0$
		\State $v[k] \gets \emptyset$
		\For{$i$ in $(0, ..., k-1)$}
		\For{$j$ in $(0, ..., n)$}
		\State $v[i]$ += $\frac{k-1}{n+1}(\frac{i+1}{n+1})^{k-2}M[k,i]$
		\EndFor
		\If{$v[i] > v[k_{max}]$}
		\State $k_{max} \gets i$
		\EndIf	
		\EndFor
		\State return $v[k_{max}]$
	\end{algorithmic}
\end{algorithm}\\
\newpage

\section*{Exercise 2}
\subsection*{First problem}
Given a weighted tree T with n nodes, find the complete graph G of minimum weight such that T $\subseteq$ G and T is the unique minimum spanning tree of G. \\
\textbf{Idea}: 
Insert edges that are not in the tree so as to obtain the complete graph. These edges must have a greater weight than those of the tree, so that T is the only MST of G. \\
\textbf{Hint}: Since \textit{Kruskal}'s algorithm uses the \textit{Union-Find} structures for representing the cuts, we use this structures to solve the problem. \\
\subsubsection*{Solution}
\begin{algorithm}
	\caption{Find complete graph}\label{euclid}
	\begin{algorithmic}[1]
		\For{$u$ in $set_1$}
			\For{$v$ in $set_2$}
				\If {$e_{new} = (u,v)$ not in $E$}
					\State $e_{new}$.setWeight($e$.getWeight() + 1)
					\State $G$.addEdge($e_{new}$)
				\EndIf
			\EndFor		
		\EndFor
	\end{algorithmic}
\end{algorithm}
Cost: \textbf{O}($|E|$), but $|E| = \frac{|V|(|V| - 1)}{2}$ and $|V| = n$, so $|E| = \frac{n^2 - n}{2}$ and the cost is \textbf{O}($n^2$)
\subsection*{Second problem}

\subsubsection*{Solution}

\newpage

\section*{Exercise 3}
\subsection*{First Problem}
\subsubsection*{Solution}
\begin {tikzpicture}
\begin{scope}[auto=left,every node/.style={circle,draw, minimum width = 1 cm}]
\node(f) at (1,2.5) {f};
\node(a) at (5,5) {a};
\node(b) at (5,0) {b};
\node(c) at (10,5) {c};
\node(d) at (10,0) {d};
\node(s) at (15,2.5) {s};
\end{scope}
\begin{scope}[every edge/.style={draw=black,very thick}]
\path [->] (f) edge node[sloped,above]{$n_a$} (a);
\path [->] (f) edge node[sloped,above]{$n_b$} (b);

\path [->] (a) edge node[sloped,above]{$n_a - s_a$} (c);
\path [->] (a) edge node[pos=0.2,sloped,above]{$n_a - s_a$} (d);
\path [->] (b) edge node[pos=0.2,sloped,above]{$n_b - s_b$} (c);
\path [->] (b) edge node[sloped,above]{$n_b - s_b$} (d);

\path [->] (a) edge[left] node{$n_a - s_a$} (b);
\path [->] (d) edge[right] node{$n_d - s_d$} (c);

\path [->] (a) edge[bend left=40] node[sloped,above]{$s_a$} (s);
\path [->] (b) edge[bend right=40] node[sloped,below]{$s_b$} (s);
\path [->] (c) edge node[sloped,above]{$s_c$} (s);
\path [->] (d) edge node[sloped,above]{$s_d$} (s);
\end{scope}
\end{tikzpicture}
\subsection*{Second Problem}
\subsubsection*{Solution}
\newpage

\section*{Exercise 4}
\subsection*{Problem}
\subsubsection*{Solution}

\end{document}