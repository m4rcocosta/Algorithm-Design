\documentclass[11pt]{article}
\usepackage{geometry}
\geometry{a4paper,top=2cm,bottom=2cm,left=2cm,right=2cm}
\usepackage[english]{babel}
\usepackage{graphicx}
\usepackage{fancyhdr}
\usepackage{lineno}
\usepackage{amsmath}
\usepackage{amssymb}
\usepackage{algorithm}
\usepackage[noend]{algpseudocode}
\usepackage {tikz}
\usetikzlibrary {positioning}
\usepackage {xcolor}
\usepackage{titlesec}
\usepackage{varwidth}

\makeatletter
\def\BState{\State\hskip-\ALG@thistlm}
\makeatother

\title{\textbf{Algorithm Design - Homework 1} \\ \bigskip \large \textbf{Sapienza University of Rome}}
\date{\textbf{\today}}
\author{\textbf{Marco Costa, 1691388}}
\pagestyle{fancy}
\fancyhead[L]{Marco Costa, 1691388}
\fancyhead[R]{Algorithm Design - Homework 1}

%section, subsection: space - left, before, after
\titlespacing{\section}{0em}{0em}{0em}
\titlespacing{\subsection}{0em}{0.5em}{0em}
\titlespacing{\subsubsection}{0em}{0.5em}{0em}
\setlength\parindent{0pt}

\begin{document}
\maketitle
\newpage

\section*{Exercise 1}
\subsection*{Problem}
The problem consists in finding the expected optimal reward playing the game ``\textit{Open the Boxes and keep the Best!}''. The input is composed by the number of boxes $k$, the number of different rewards $n$ an
the cost $c_i$ of every box. Two algorithms are required. The first one's complexity must be \textbf{\textit{O}}($n^2 \cdot k$), while the second one's complexity must be \textbf{\textit{O}}($n \cdot k$)
\subsubsection*{Solution}
In order to solve the problem, we can define a matrix \textit{M} composed of $k$ rows and $n$ columns, where the element $M[i, j]$ represents the expected value by opening the box i and having already as maximum prize j. \\
To populate the matrix we can use \textbf{backward induction}, proceeding first considering the last box and choosing what to do for all the possible prizes. Using this information, we can then determine what to do at the penultimate box. This process continues backwards until we have determined the best action for every possible prize for each box. So the element $\mathbf{M[0, 0]}$ is the expected optimal reward.\\
Let's explain the mathematical formula: \\
- $j \cdot \frac{j+1}{n+1}$ (or $M[i+1, j] \cdot \frac{j+1}{n+1}$) is the probability of obtaining a reward $r \le j$; \\
- $\frac{1}{n+1}\sum\limits_{h = j + 1}^n{h}$ (or $\frac{1}{n+1}\sum\limits_{h = j + 1}^n{M[i+1, j]}$) is the probability of obtaining a reward $r > j$; \\
- $c[i]$ is the cost to pay opening the $i^{th}$ box. \\
To improve the complexity to \textbf{\textit{O}}($n \cdot k$) we can eliminate the summations, since in the worst case we have a sum over $n$ (\textbf{\textit{O}}($n$)) in nested cycles (\textbf{\textit{O}}($n \cdot k$)) having a total cost of\textbf{\textit{O}}($n^2 \cdot k$). To do this we must distinguish the two cases:
\begin{itemize}
	\item \textbf{\textit{Last box}}: we can remove this summation using a mathematical trick: \\
	$\sum\limits_{i = j + 1}^n{i} = \sum\limits_{i = 0}^n{i} - \sum\limits_{i = 0}^j{i} = \frac{n \cdot (n+1)}{2} - \frac{j \cdot (j+1)}{2} = \frac{n \cdot (n+1) - j \cdot (j+1)}{2}$;
	\item \textbf{\textit{Other boxes}}: we can use an auxiliary vector in which we keep track of the sum of the elements of the matrix related to the box next to the current one.
\end{itemize}

Here there are the two \textbf{\textit{Algorithms}}:

\begin{minipage}[t]{0.49\textwidth}
\begin{algorithm}[H]
	\caption{Get optimal expected value (\textbf{\textit{O}}($n^2 \cdot k$))}\label{euclid}
	\begin{algorithmic}[1]
		\State $M[k, n+1] \gets 0$
		\For{$i$ in $(k-1, ..., 0)$}
		\For{$j$ in $(n+1, ..., 0)$}
		\State $r \gets 0$
		\If{$i = k-1$}
		\State $r \gets j \cdot \frac{j+1}{n+1} + \frac{1}{n+1}\sum\limits_{h = j + 1}^n{h} - c[i]$
		\Else
		\State \begin{varwidth}[t]{\linewidth} $r \gets M[i+1, j] \cdot \frac{j+1}{n+1} + \frac{1}{n+1} \cdot$ \par
			\hskip\algorithmicindent $\sum\limits_{h = j + 1}^n{M[i+1, h]} - c[i]$
		\end{varwidth}
		\EndIf
		\State $M[i, j] \gets \max(j, r)$
		\EndFor		
		\EndFor
		\State \Return $M[0, 0]$
	\end{algorithmic}
\end{algorithm}
\end{minipage}
\hfill
\begin{minipage}[t]{0.49\textwidth}
\begin{algorithm}[H]
	\caption{Get optimal expected value (\textbf{\textit{O}}($n \cdot k$))}\label{euclid}
	\begin{algorithmic}[1]
		\State $M[k, n+1] \gets 0$
		\State $sums[n+1] \gets 0$
		\For{$i$ in $(k-1, ..., 0)$}
		\For{$j$ in $(n+1, ..., 0)$}
		\State $r \gets 0$
		\If{$i = k-1$}
			\State $r \gets j \cdot \frac{j+1}{n+1} + \frac{1}{n+1} \cdot \frac{n \cdot (n+1)-j \cdot (j+1)}{2} - c[i]$
		\Else
			\State $r \gets M[i+1, j] \cdot \frac{j+1}{n+1} + \frac{sums[j]}{n+1} - c[i]$
		\EndIf
		\State $M[i, j] \gets \max(j, r)$
		\If{$j = n$}
			\State $sums[j] \gets 0$
		\Else
			\State $sums[j] \gets sums[j+1] + M[i, j+1]$
		\EndIf
		\EndFor		
		\EndFor
		\State \Return $M[0, 0]$
	\end{algorithmic}
\end{algorithm}
\end{minipage}
\newpage

\section*{Exercise 2}
\subsection*{First problem}
The problem is to the complete graph \textbf{\textit{G}} of minimum weight given a weighted tree \textbf{\textit{T}}, such that  \textbf{\textit{T}} is the unique minimum spanning tree of \textbf{\textit{G}}. Algorithm's run time must be polynomial in $n$.
\subsubsection*{Solution}
Insert edges that are not in the tree so as to obtain the complete graph. These edges must have a greater weight than those of the tree, so that \textbf{\textit{T}} is the only \textbf{\textit{MST}} of \textbf{\textit{G}}. Since \textit{Kruskal}'s algorithm uses the \textit{Union-Find} structures for representing the cuts, we use this structures to solve the problem. \\
Let's define with \textbf{\textit{V}} the set of nodes of \textbf{\textit{T}} and with \textbf{\textit{E}} the set of edges of \textbf{\textit{T}}.
\begin{algorithm}
	\caption{Find complete graph}\label{euclid}
	\begin{algorithmic}[1]
		\For{$v \in V$} \qquad \qquad \qquad \qquad \qquad \qquad \qquad \qquad \qquad \qquad \qquad \qquad (\textbf{\textit{O}}($n$))
			\State $v.initializeUnionFindSingleton()$
		\EndFor
		\State $T.sortEdgesByAscendingWeights()$ \qquad \qquad \qquad \qquad \qquad \qquad \qquad (\textbf{\textit{O}}($n\log{n}$))
		\State $G \gets \emptyset$
		\For {$e \gets (v_1, v_2) \in E$} \qquad \qquad \qquad \qquad \qquad \qquad \qquad \qquad \qquad \qquad (\textbf{\textit{O}}($n^3$))
			\State  $G.addEdge(e)$
			\State  $set_1 \gets v_1.findComponents()$
			\State  $set_2 \gets v_2.findComponents()$
			\For {$u \in set_1$}
				\For {$v \in set_2$}
					\State $\hat{e} \gets (u, v)$
					\If {$\hat{e} \notin E$}
						\State $\hat{e}.setWeight(e.getWeight() + 1)$
						\State $G.addEdge(\hat{e})$
					\EndIf
				\EndFor
			\EndFor
			\State $union(set_1, set_2)$
		\EndFor
		\State \Return $G$
	\end{algorithmic}
\end{algorithm} \\
\textbf{Cost}: \textbf{\textit{O}}($|V|\log{|V|} + |E| \cdot |V|^2$) = \textbf{\textit{O}}($n\log{n} + (n-1) \cdot n^2$) = \textbf{\textit{O}}($n\log{n} + n^3 - n^2$) = \textbf{\textit{O}}($n^3$), given by the three nested cycles over $n$.
\subsection*{Second problem}
Find the total weight of the complete graph \textbf{\textit{G}} of minimum weight given a weighted tree \textbf{\textit{T}}, such that  \textbf{\textit{T}} is the unique minimum spanning tree of \textbf{\textit{G}}. Algorithm's complexity must be \textbf{\textit{O}}($n \log{n}$).
\subsubsection*{Solution}
This problem is quite similar to the previous one, but in this case there is no need to create all the edges of the graph.
\begin{algorithm}
	\caption{Find weight of the complete graph}\label{euclid}
	\begin{algorithmic}[1]
		\For{$v \in V$} \qquad \qquad \qquad \qquad \qquad \qquad \qquad \qquad \qquad \qquad \qquad \qquad (\textbf{\textit{O}}($n$))
			\State $v.initializeUnionFindSingleton()$
		\EndFor
		\State $T.sortEdgesByAscendingWeights()$ \qquad \qquad \qquad \qquad \qquad \qquad \qquad (\textbf{\textit{O}}($n\log{n}$))
		\State $w_{total} \gets 0$
		\For {$e \gets (v_1, v_2) \in E$} \qquad \qquad \qquad \qquad \qquad \qquad \qquad \qquad \qquad \qquad (\textbf{\textit{O}}($n \log{n}$))
			\State $w_{total}$ \textit{+=} $e.getWeight()$
			\State  $set_1 \gets v_1.findComponents()$
			\State  $set_2 \gets v_2.findComponents()$
			\State $w_{total}$ \textit{+=} $(set_1.size() \times set_2.size() - 1) \times (e.getWeight() + 1)$
			\State $union(set_1, set_2)$
		\EndFor
		\State \Return $w_{total}$
	\end{algorithmic}
\end{algorithm} \\
\textbf{Cost}: \textbf{\textit{O}}($|V|\log{|V|} + |E| \log{|V|}$) = \textbf{\textit{O}}($n\log{n} + n\log{n}$) = \textbf{\textit{O}}($n\log{n}$) (\textit{union} costs \textbf{\textit{O}}($\log{n}$)).
\newpage

\section*{Exercise 3}
\subsection*{First Problem}
The problem consists in modeling Federico's business as a flow problem in a graph \textbf{\textit{G}}, only consisting of regular vertices, one source, one sink, and capacitated edges to find out how many chocolates he should make every week. An example with $|F| = 4$ is required.
\subsubsection*{Solution}
\textbf{Example}:

\begin {tikzpicture}
\begin{scope}[auto=left,every node/.style={circle,draw, minimum width = 1 cm}]
\node(f) at (0,0) {F};
\node(ma) at (3,1) {m$_a$};
\node(a) at (6.5,1) {f$_a$};
\node(mb) at (3,-1) {m$_b$};
\node(b) at (6.5,-1) {f$_b$};
\node(mc) at (9.5,1) {m$_c$};
\node(c) at (13,1) {f$_c$};
\node(md) at (9.5,-1) {m$_d$};
\node(d) at (13,-1) {f$_d$};
\node(s) at (16,0) {s};
\end{scope}
\begin{scope}[every edge/.style={draw=black,very thick}]
\path [->] (f) edge node[sloped,above]{$n_1$} (ma);
\path [->] (f) edge node[sloped,above]{$n_2$} (mb);

\path [->] (a) edge node[sloped,below]{$n_a - s_a$} (mc);
\path [->] (b) edge node[sloped,above]{$n_b - s_b$} (md);
\path [->] (c) edge[bend right=20] node[sloped,above]{$n_c - s_c$} (ma);
\path [->] (d) edge[bend left=20] node[sloped,below]{$n_d - s_d$} (mb);

\path [->] (a) edge[sloped,above,pos=0.75] node{$n_a - s_a$} (mb);
\path [->] (d) edge[sloped,above,pos=0.2] node{$n_d - s_d$} (mc);
\path [->] (b) edge[sloped,above,pos=0.2] node{$n_b - s_b$} (ma);
\path [->] (c) edge[sloped,above,pos=0.75] node{$n_c - s_c$} (md);

\path [->] (ma) edge node[sloped,above]{$n_a$}(a);
\path [->] (mb) edge node[sloped,above]{$n_b$}(b);
\path [->] (mc) edge node[sloped,above]{$n_c$}(c);
\path [->] (md) edge node[sloped,above]{$n_d$}(d);

\path [->] (a) edge[bend left=30] node[sloped,above]{$s_a$} (s);
\path [->] (b) edge[bend right=30] node[sloped,below]{$s_b$} (s);
\path [->] (c) edge node[sloped,above]{$s_c$} (s);
\path [->] (d) edge node[sloped,above]{$s_d$} (s);
\end{scope}
\end{tikzpicture}
\subsection*{Second Problem}
Adjust the network knowing that each friend will have an associated building $b_f \in B$ and find out if and how this
impacts Federico’s business.
\subsubsection*{Solution}
\begin {tikzpicture}
\begin{scope}[auto=left,every node/.style={circle,draw, minimum width = 1 cm}]
	\node(f) at (0,0) {F};
	\node(ma) at (2,1) {m$_a$};
	\node(a) at (5.8,1) {f$_a$};
	\node(mb) at (2,-1) {m$_b$};
	\node(b) at (5.8,-1) {f$_b$};
	\node(mc) at (8.2,1) {m$_c$};
	\node(c) at (12,1) {f$_c$};
	\node(md) at (8.2,-1) {m$_d$};
	\node(d) at (12,-1) {f$_d$};
	\node(b1) at (14,2) {$b_1$};
	\node(b2) at (14,0) {$b_2$};
	\node(b3) at (14,-2) {$b_3$};
	\node(s) at (16,0) {s};
\end{scope}
\begin{scope}[every edge/.style={draw=black,very thick}]
	\path [->] (f) edge node[sloped,above]{$n_1$} (ma);
	\path [->] (f) edge node[sloped,above]{$n_2$} (mb);
	
	\path [->] (a) edge node[sloped,below]{$n_a - s_a$} (mc);
	\path [->] (b) edge node[sloped,above]{$n_b - s_b$} (md);
	\path [->] (c) edge[bend right=20] node[sloped,above]{$n_c - s_c$} (ma);
	\path [->] (d) edge[bend left=20] node[sloped,below]{$n_d - s_d$} (mb);
	
	\path [->] (a) edge[sloped,above,pos=0.75] node{$n_a - s_a$} (mb);
	\path [->] (d) edge[sloped,above,pos=0.2] node{$n_d - s_d$} (mc);
	\path [->] (b) edge[sloped,above,pos=0.2] node{$n_b - s_b$} (ma);
	\path [->] (c) edge[sloped,above,pos=0.75] node{$n_c - s_c$} (md);
	
	\path [->] (ma) edge node[sloped,above]{$n_a$}(a);
	\path [->] (mb) edge node[sloped,above]{$n_b$}(b);
	\path [->] (mc) edge node[sloped,above]{$n_c$}(c);
	\path [->] (md) edge node[sloped,above]{$n_d$}(d);
	
	\path [->] (a) edge[bend left=10] node[sloped,above]{$s_a$} (b1);
	\path [->] (b) edge[bend right=10] node[sloped,below]{$s_b$} (b3);
	\path [->] (c) edge node[sloped,above]{$s_c$} (b1);
	\path [->] (d) edge node[sloped,above]{$s_d$} (b2);
	
	\path [->] (b1) edge node[sloped,above]{$c_1$} (s);
	\path [->] (b2) edge node[sloped,above]{$c_2$} (s);
	\path [->] (b3) edge node[sloped,above]{$c_3$} (s);
\end{scope}
\end{tikzpicture}
\newpage

\section*{Exercise 4}
\subsection*{Problem}
The problem consists in showing that it is not possible to solve the problem of scheduling $n$ tasks with quick algorithm. Each job  $j$ of the $n$ jobs has an earliest time $s_j$ to start task, a deadline $d_j$ when it has to be finished, and the length $l_j \in \mathbb{N}$ that specifies the time needed to complete it. There is also an upper bound time $k$ and $\sum\limits_{j \in J}l_j \le k$.
\subsubsection*{Solution}
In order to prove that this problem is not solvable with quick algorithm we need to prove that this problem is \textbf{\textit{NP-Hard}}. We can do it reducing another \textit{NP-Hard} problem to the current problem. Let's define our problem as $\Theta$, we need to find a well known \textit{NP-Hard} problem $\Omega$ such that:
\begin{itemize}
	\item $\omega \in \Omega \implies  \theta \in \Theta$
	\item $\theta \in \Theta \implies  \omega \in \Omega$ (or equivalently $\neg\omega \in \Omega \implies  \neg\theta \in \Theta$)
\end{itemize}
\textbf{\underline{Observation}}: The earliest start time  $s_j$ indicate only at what time the job $j$ is available (it does not need to be done at that moment, but it does have to be done before its deadline $d_j$). \\ \\
\textbf{\underline{Observation}}: Our scheduling problem is solvable if and only if each job $j$ meets its deadline $d_j$ and $\sum\limits_{j \in J}l_j \le k$. \\
We can use \textbf{\textit{Subset Sum}} for our scope ($\Omega$ = \textit{Subset Sum}). \\
\textbf{Subset sum}: Given a set of positive integers and an integer $k$, is there any non-empty subset whose sum to $s$.
\begin{enumerate}
	\item Prove that $\omega \in \Omega \implies  \theta \in \Theta$: \\
	Let's suppose we start with a solvable instance of \textit{Subset Sum} Problem. Given $X = \{x_1, ..., x_n\}$ a set of positive integers to which there is a subset $X'$ such that its elements add up to $k$ and the sum of the whole set is $\lambda$ ($\sum\limits_{x_i \in X'}x_i = k$ and $\sum\limits_{x_i \in X}x_i = \lambda$). Let's define jobs $\{j_1, ..., j_n\}$ such that $s_i$, $l_i = x_i$ and $d_i = t+1$ for every $j_i$ (earliest start time, duration of the job and deadline respectively). This instance solves the scheduling problem because it's possible to arrange the jobs in any order and they will always meet their deadline.
		
	\item Prove that $\theta \in \Theta \implies  \omega \in \Omega$: \\
	Let's suppose we start with a solvable instance of our \textit{Scheduling} problem. Given the set of jobs $\{j_1, ..., j_n, j_{n+1}\}$ where the first $n$ jobs are defined the same as before and job $j_{n+1}$ having $s_{n+1} = k$, $l_{n+1} = 1$ and $d_{n+1} = k+1$. The extra job must be done in $[k, k+1]$, but there would still be $\lambda$ units of time available in the time interval $[0, \lambda+1]$. Since this instance is solvable, we have $|X'|$ processes that can be executed before job $j_{n+1}$ and another sequence of jobs after. The set $|X'|$ solves the \textit{Subset Sum} problem.
\end{enumerate}
We proved that our scheduling problem is \textit{NP-Hard}, so we can say that it is not solvable with a quick algorithm.

\end{document}
